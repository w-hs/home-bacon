\section{Besonderheiten bei der Entwicklung einer Android-Wear-App}
In diesem Kapitel gehen wir auf Besonderheiten, die bei der Entwicklung einer Anwendung für die Smartwatch bestehen, ein. Zunächst erklären wir kurz was die grundsätzlichen Probleme bzw. Herausforderungen sind. Anschließend gehen wir ganz konkret auf unsere Applikation und ihre Besonderheiten ein.

\subsection{Einschränkungen durch die Smartwatch}
Grundsätzlich verfügen Smartwatches über die gleichen Eingabemöglichkeiten wie Smartphones. Durch das kleine Display entfällt jedoch die Möglichkeit einer vollständigen Tastatur, so dass Benutzerinteraktionen häufig über Wischgestiken oder sogenannte "`Action Buttons"'\footnote{Action Buttons sind vom System bereitgestellte Buttons mit einem Icon und einen beschreibenden Text. Sie werden im Vollbild angezeigt und dienen der Annahme von Benutzerinteraktionen.} realisiert werden. 

Eine weitere Besonderheit ist neben der Größe auch die Form des Displays. Damit die Applikation auf Uhren mit eckigen und rundem Display jeweils gleich angezeigt werden bietet Android-Wear ein spezielles "`BoxInsetLayout"', dass auch auf einem runden Display immer den größtmöglichen eckigen Bereich zur Anzeige nutzt.

Darüber hinaus besitzen nicht alle Smartwatches die Möglichkeit sich mit einem WLAN zu verbinden. Daher ist ein direkter Zugriff auf das Internet aus der Smartwatch-App nicht immer möglich.

\subsection{HomeBacon-App}
Ziel unserer Smartwatch-App ist es eine bestimmte Menge von Notizen zur Anzeige zu bringen. Daher haben wir uns im ersten Schritt die von Android-Wear bereitgestellten UI-Elemente angesehen. Unter den verschiedenen Möglichkeiten Notizen, bestehend aus Titel und einem Text, darzustellen, erschien uns die Darstellung in Form von Cards\footnote{Eine Übersicht der Android-Wear-UI-Patterns befindet sich unter \url{http://developer.android.com/design/wear/patterns.html}} am geeignetsten.

Um die Anzeige mehrerer Notizen auf dem kleinem Display der Smartwatch zu realisieren, haben wir uns überlegt die Cards horizontal anzuordnen. Eine solche Anordnung lässt sich als "`2D Picker"' mithilfe eines GridViewPagers realisieren. Wie unter Android üblich mussten wir dazu unterschiedliche Klassen ableiten, Methoden überschreiben und aus programmieren. Über einen Button, der in Form eines Mülleimers dargestellt wird, lassen sich bestehende Notizen löschen. Ein Screenshot der prototypischen Anwendung findet sich unterhalb des Ergebnisses. TODO!

Ursprünglich hatten wir geplant, dass es möglich sein sollte eine Messung über die Uhr durchzuführen. Den entsprechenden Dialog hatten wir als weitere Card geplant, die über eine vertikale Wischgeste aufrufbar sein sollte. Das Problem war allerdings diese Funktionalität im Rahmen des 2D Pickers zu realisieren, da jede weitere Zeile gleich viele Elemente enthalten muss. In unserem Fall hätten wir dazu pro Notiz einen neuen Einstellungsdialog instantiieren müssen. Daher haben wir uns dazu entschieden diese Funktionalität auf dem Smartphone zu implementieren.




