\section{Besonderheiten bei der Entwicklung einer Android-Wear-App}
In diesem Kapitel gehen wir auf Besonderheiten, die bei der Entwicklung einer Anwendung für die Smartwatch bestehen, ein. Zunächst erklären wir kurz was die grundsätzlichen Probleme bzw. Herausforderungen sind. Anschließend gehen wir ganz konkret auf unsere Applikation und ihre Besonderheiten ein.

\subsection{Einschränkungen durch die Smartwatch}
Grundsätzlich verfügen Smartwatches über die gleichen Eingabemöglichkeiten wie Smartphones. Durch das kleine Display entfällt jedoch die Möglichkeit einer vollständigen Tastatur, so dass Benutzerinteraktionen häufig über Wischgestiken oder sogenannte "`Action Buttons"'\footnote{Action Buttons sind vom System bereitgestellte Buttons mit einem Icon und einen beschreibenden Text. Sie werden im Vollbild angezeigt und dienen der Annahme von Benutzerinteraktionen.} realisiert werden. 

Eine weitere Besonderheit ist neben der Größe auch die Form des Displays. Damit die Applikation auf Uhren mit eckigen und rundem Display jeweils gleich angezeigt werden bietet Android-Wear ein spezielles "`BoxInsetLayout"', dass auch auf einem runden Display immer den größtmöglichen eckigen Bereich zur Anzeige nutzt.

Darüber hinaus besitzen nicht alle Smartwatches die Möglichkeit sich mit einem WLAN zu verbinden. Daher ist ein direkter Zugriff auf das Internet aus der Smartwatch-App nicht immer möglich.

\subsection{HomeBacon-App}
Ziel unserer Smartwatch-App ist es eine bestimmte Menge von Notizen zur Anzeige zu bringen. Daher haben wir uns im ersten Schritt die von Android-Wear bereitgestellten UI-Elemente angesehen. Unter den verschiedenen Möglichkeiten Notizen, bestehend aus Titel und einem Text, darzustellen, erschien uns die Darstellung in Form von Cards\footnote{Eine Übersicht der Android-Wear-UI-Patterns befindet sich unter \url{http://developer.android.com/design/wear/patterns.html}} am geeignetsten.

Um die Anzeige mehrerer Notizen auf dem kleinem Display der Smartwatch zu realisieren, haben wir uns überlegt die Cards horizontal anzuordnen. Eine solche Anordnung lässt sich als "`2D Picker"' mithilfe eines GridViewPagers realisieren. 


Im nächsten Schritt benötigten wir eine Möglichkeit mehrere Notizen, also mehrere Cards anzuzeigen. Wir waren uns von Beginn an einig, dass die NAvigation durch die Cards über Wischgestiken realisiert werden sollen. Auch das Löschen von Cards hatten wir uns ursprünglich als Wischgestik vorgestellt. Dabei schwebte uns vor, dass man über Rechts- bzw. Links wischen durch eine Reihe von Notizen durch wischen kann und über einen Wischgestik nach unten oder oben die Notiz wieder entfernen kann.

Auf der Suche nach einer Möglichkeit mehrere Karten in einer Reihe anzuzeigen fanden wir lediglich den sogenannten "`2D-Picker"', dieser bietet die Möglichkeit verschiedene Karten als Matrix anzuordnen.

	- unkomfortabel, viel Komplexität, große Hürde
	- Unsere Implementierung des 2D-Pickers... 
		Komplette Matrix war nötig
		Keine Möglichkeit null zurückzugeben
		Ursprüngliche Idee Einstellungskarte zu machen
		Dann Entscheidung alle Eingaben aufs Smartphone zu legen
		So dass Uhr nur zur Anzeige bzw. Löschen von Notizen geht
		Smartphone angenehmere Bedienung, Internetverbindung... Uhr WLAN mal kein WLAN?
