\section{Besonderheiten bei der Entwicklung einer Android-Wear-App}
In diesem Kapitel gehen wir auf Besonderheiten, die bei der Entwicklung einer Anwendung für die Smartwatch bestehen, ein. Zunächst erklären wir kurz was die grundsätzlichen Probleme bzw. Herausforderungen sind. Anschließend gehen wir ganz konkret auf unsere Applikation und ihre Besonderheiten ein.

\subsection{Grundsätzliches}
Grundsätzlich dient die Smartwatch sozusagen als verlängertes Display des Smartphones. Damit verfügt die Smartwatch über nur wenige Eingabemöglichkeiten. Die meisten Anforderungen werden sind über spezielle Wischgestiken realisiert. Eine Tastatur gibt es nicht. Bestimmte Aktionen werden in der Regel über "`Action Buttons"' (s. http://developer.android.com/design/wear/patterns.html) realisiert. Hinzukommt, dass manche Uhren eckig und manche Uhren rund sind. Hierfür bietet Android Wear eine spezielle Komponente das "`BoxInsetLayout"' (s. http://developer.android.com/training/wearables/ui/layouts.html), diese Komponente sorgt dafür, dass auch auf runden Display immer ein größtmöglicher ekciger Bereich zu Verfügung steht.

\subsection{Home-Bacon-Wear-App}


Konkreter
	- Lange überlegt wie wir Notizen richtig zur Anzeige bringen
		- Inkl. der Möglichkeit zum Löschen
	- Soviele UIs bietet Android-Wear gar nicht, Cards / Fragments / 2D Picker...
	- 

- Lange auf der Suche nach dem Richtigen Layout / Controls / Widgets
- Offenbar keine Eingabemöglichkeiten außer Touchen und Wischen
- Problem: Runde und Eckige Displays
- Insgesammt alles sehr klein

- Richtige Anzeige der Notizen...
- Verweise auf Android Developers...