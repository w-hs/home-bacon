\section{Besonderheiten bei der Entwicklung einer Android-Wear-App}
In diesem Kapitel gehen wir auf Besonderheiten, die bei der Entwicklung einer Anwendung für die Smartwatch bestehen, ein. Zunächst erklären wir kurz was die grundsätzlichen Probleme bzw. Herausforderungen sind. Anschließend gehen wir ganz konkret auf unsere Applikation und ihre Besonderheiten ein.

\subsection{Einschränkungen durch die Smartwatch}
Grundsätzlich verfügen Smartwatches über die gleichen Eingabemöglichkeiten wie Smartphones. Durch das kleine Display entfällt jedoch die Möglichkeit einer vollständigen Tastatur, so dass Benutzerinteraktionen häufig über Wischgestiken oder sogenannte "`Action Buttons"'\footnote{Action Buttons sind vom System bereitgestellte Buttons mit einem Icon und einen beschreibenden Text. Sie werden im Vollbild angezeigt und dienen der Annahme von Benutzerinteraktionen.} realisiert werden. 

Eine weitere Besonderheit ist neben der Größe auch die Form des Displays. Damit die Applikation auf Uhren mit eckigen und rundem Display jeweils gleich angezeigt werden bietet Android-Wear ein spezielles "`BoxInsetLayout"', dass auch auf einem runden Display immer den größtmöglichen eckigen Bereich zur Anzeige nutzt.

\subsection{Home-Bacon-Wear-App}


Konkreter
	- Lange überlegt wie wir Notizen richtig zur Anzeige bringen
		- Inkl. der Möglichkeit zum Löschen
	- Soviele UIs bietet Android-Wear gar nicht, Cards / Fragments / 2D Picker...
	- 

- Lange auf der Suche nach dem Richtigen Layout / Controls / Widgets
- Offenbar keine Eingabemöglichkeiten außer Touchen und Wischen
- Problem: Runde und Eckige Displays
- Insgesammt alles sehr klein

- Richtige Anzeige der Notizen...
- Verweise auf Android Developers...