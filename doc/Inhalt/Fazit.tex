\section{Fazit}
Zu Beginn des Projekts haben wir unsere Idee spezifiziert und Hardwareevaluationen durchgeführt. Als Ergebnis bestellten wir acht Bluetooth-Beacons, die den Low-Energy Standard erfüllen, und zwei Smartwatches aus stark unterschiedlichen Preissegmenten, um deren Ausstattung und Qualität beurteilen zu können.

Nachdem wir uns einige Zeit mir den Uhren auseinandergesetzt hatten, stellten wir gravierende Unterschiede hinsichtlich der Qualität fest. Die LG Watch Urbane wirkte deutlich hochwertiger, als das günstigere Modell SmartWatch 3 von Sony, welches beispielsweise nicht einmal in der Lage war ein klares Weiß anzuzeigen. Darüber hinaus war die Akkulaufzeit, die laut Hersteller zwei Tage betragen soll, mit realistischen sechs Stunden verglichen zum teuren Modell mit einem Tag eine Zumutung. Insgesamt hatten wir den Eindruck, dass diese Generation der Smartwatch noch nicht ganz ausgereift ist. 

Gerade zu Beginn des Projekts haben wir lange nach sinnvollen Anwendungsfällen gesucht. Wir sehen heute einen möglichen Bedarf im Bereich der Heimautomatisierung. Unser implementiertes Tracking der Person erlaubt beispielsweise, dass Musik, Licht, Duft oder Ähnliches der Person quasi folgen bzw. sogar "`vorauslaufen"'.

Da bisher noch keiner von uns für die Plattform Android-Wear entwickelt hatte, haben wir im Laufe des Projekts eine Menge gelernt. Insbesondere bei der Kommunikation zwischen Smartphone und Smartwatch haben wir die vorhandenen Wear-APIs intensiv ausprobiert. Außerdem lernten wir die grundlegenden Oberflächenkomponenten der Android-Wear-Plattform kennen. Besonders erfreulich war darüber hinaus, dass wir das Wissen aus anderen Veranstaltungen bei der Implementierung nutzen konnten. Gerade beim maschinellen Lernen, das wir zur automatischen Raumerkennung verwenden, konnten wir auf Erfahrungen aus den Modulen Fortgeschrittene Datenbanktechniken und Artificial Intelligence zurückgreifen.

Der in dieser Ausarbeitung beschriebene Prototyp verkörpert unseren Erfolg mit der Raumerkennung mittels Bluetooth Low Energy und Wearables. Wir nutzen eine Smartwatch und einmalig zur initialen Einrichtung ein Smartphone zur Bestimmung des Raumes, in dem sich der Träger befindet. Um einen simplen Anwendungsfall um die Raumerkennung herum zu haben, implementierten wir das Hinterlassen von virtuellen Notizen in Räumen und an Raumübergängen. Während der Umsetzung hatten wir das Mobile Dilemma stets im Gedächtnis, sodass die Smartwatch nach der Einrichtung vollkommen autark ohne jegliche Internet- oder Smartphoneverbindung mit ressourcenschonenden Algorithmen die Raumerkennung meistert.

Dieses tolle Ergebnis haben wir in der kurzen Zeit aufgrund unseres genauen Projektplans, den wir zu Beginn des Projekts aufgestellt haben, erreicht. Rückblickend können wir sagen, dass wir diesen, mit einer gravierenden Ausnahme, strikt eingehalten haben. Es war ursprünglich angedacht ebenfalls ein Gegenstandstracking zu implementieren, mit dem es möglich sein sollte, Gegenstände auf einen Raum genau zu lokalisieren. Dieses Feature konnten wir am Projektende aufgrund mangelnder Zeit leider nicht mehr fertigstellen. Bei unserer Retrospektive haben wir den Grund darin ausgemacht, dass unsere Meilensteine viel zu allgemein formuliert waren, wie z.~B. “Smartphone-App fertig”. Wir haben schlichtweg vergessen, dass wir das Gegenstandstracking implementieren wollten, und wurden durch unseren Projektplan nicht daran erinnert. Aus dieser Erfahrung lernen wir, zukünftige Projekt-Arbeitspakete und Meilensteine genauer zu spezifizieren und mit den gewünschten Akzeptanzkriterien zu verknüpfen. Dieser Fehler hat uns so sehr geärgert, dass er uns wahrscheinlich nie wieder passieren wird.

Dessen ungeachtet haben wir mit einer super Gruppenleistung dennoch unser Hauptziel einer Raumerkennung mithilfe von Bluetooth und Android-Wear erreicht. Wir haben auch in diesem Projekt wieder vieles hinzugelernt und schauen heute mit einem weinenden Auge auf die zu Ende gehende Studienzeit und mit einem lachenden auf all die Herausforderungen, die draußen auf uns warten. Wir fühlen uns gut gewappnet, nicht zuletzt dank Ihnen. Vielen Dank!
