\section{Fazit}

Am Anfang dieses Projekts habe wir zuerst die Fragestellung genauer spezifiziert und erste Hardware Evaluationen vorgenommen. Diese Arbeitsschritte liefen weitgehend problemlos ab. Als Ergebnis daraus bestellten wir die fürs Projekt benötigte Hardware, bestehend aus 2 Smartwatches und 8 Bluetooth-Low-Energie-Beacons. Ein weiteres Resultat war ein Projektablauf Plan, der auch im laufe des Projekts eingehalten wurde und Anwendungsfalldiagramm, zur Strukturen und des Verhaltens unseres Prototypen. 

Im laufe der Entwicklung konnten wir große Erfahrungen im Umgang mit Android-Wear und das Entwickeln von Samrtwatch-Apps sammeln. Weiter war es uns möglich Fachübergreifendes Wissen aus Veranstaltungen wie "Artificial Intelligence - Künstliche Intelligenz" zu nutzen. Das half dabei die Raumerkennung über Bluetooth-Beacons zu realisieren.

Am Ende des Projekts hatten wir eine Android Wear Smartwatch-App, eine Android Smartphone-App und einen Web-Service. Die Smartwatch-App dient zu Darstellung von Notizen und zum messen der Bluetooth-Beacons. Die Samrtwatch-App startet und stoppt eine Messung, sowie ermöglicht sie das erstellen von Notizen und sendet und empfängt Daten vom Web-Service. Der Web-Service erhält Messungen und berechnet daraus das Model zur Raumbestimmung.

Mithilfe all dieser Komponenten ist es dann möglich eine Automatisierte Raumerkennung zu machen und virtuelle Notizen in Räume zu legen bzw. Notizen beim verlassen oder betreten eines Raumes anzuzeigen.