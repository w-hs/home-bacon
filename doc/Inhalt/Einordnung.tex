\section{Einordnung in Mobile-Computing}
Bei der Einordnung unseres Projektes in den Themenkomplex Mobile-Computing sind vor allem die Bereiche Ubiquitous- und Wearable-Computing relevant. 

In der heutigen Zeit besitzt nahezu jeder Mensch ein Smartphone. Die intelligenten Telefone sind allgegenwärtig und greifen permanent, von überall, auf Daten und Dienste aus dem Internet zurück. Die Zeiten in denen eine Person nur einen Computer besitzt sind vorbei. Neben dem PC verfügt man zusätzlich mindestens über ein Smartphone. Die Meisten besitzen darüber hinaus noch ein Tablet und bei vielen kommt inzwischen noch eine Smartwatch zusätzlich hinzu.

Neu am sogenannten Wearable-Computing ist, dass die Geräte aktiv beim Tragen genutzt werden. Als verlängertes Display für das Smartphone machen sie vor allem durch Ereignisse, wie beispielsweise eingehende Nachrichten, auf sich aufmerksam. 

\subsection{Art der Mobilität}
Unser Projekt beschäftigt sich hauptsächlich mit der Mobilität von Endgeräten. In unserem Szenario bewegt sich der Anwender, ausgestattet mit einer Smartwatch, durch sein Eigenheim und wird dabei aktiv über den Wechsel der Räumlichkeiten informiert. Dabei werden ihm im Vorfeld hinterlegte Notizen angezeigt, die er dann während seiner Bewegung wahrnimmt. Damit fällt unser Projekt in den Bereich der kontinuierlichen physikalischen Bewegung.

\subsection{Abstraktionen}
In unserem Modell beschreibt die Lokation den Raum, in dem sich die Smartwatch aktuell befindet. Zur Ausmessung der Räumlichkeiten ist das Smartphone erforderlich, während zur späteren Erkennung eines Raumwechsels nur die Smartwatch benötigt wird. Damit ist der Kontext lediglich davon abhängig, ob der Anwender neben seiner Smartwatch zusätzlich sein Smartphone dabei hat.

Die Smartwatch soll in der Lage sein einen Raumwechsel automatisch zu erkennen. Da die Uhr über nur wenig Rechenleistung verfügt, muss der entsprechende Algorithmus möglichst ressourcenschonend arbeiten. Daher erscheint es sinnvoll die Berechnung des mathematischen Modells in einen Webservice auszulagern und über entsprechende Middleware anzusprechen.

Um die Offline-Problematik zu umgehen, soll die Anwendung auf der Smartwatch auch ohne eine bestehende Internetverbindung in der Lage sein einen Raumwechsel zu erkennen. Eine weitere Herausforderung neben der Offline-Problematik ist, aufgrund unterschiedlicher Displayformen und -Größen, das Thema Design und Usability.

\subsection{Mobiles Dilemma}
- Kernfunktionalität auch Offline (In-Door-Lokalisation + Notizen)
\\- Online könnte die Synchronisierung mit anderen Geräten bringen


aufwendige Modellberechnungen die nicht auf der Uhr passieren sollen...

Berechnung nur einmal vornehmen, man möchte ja unabhängig vom Internet sein! Daher nur einmal Zugriff auf das Web nötig.