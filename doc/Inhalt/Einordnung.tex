\section{Einordnung in Mobile-Computing}
Bei der Einordnung unseres Projektes in den Themenkomplex Mobile-Computing sind vor allem die Bereiche Ubiquitous- und Wearable-Computing relevant. 

In der heutigen Zeit besitzt nahezu jeder Mensch ein Smartphone. Die intelligenten Telefone sind allgegenwärtig und greifen permanent von überall auf Daten und Dienste aus dem Internet zurück. Die Zeiten in denen eine Person nur einen Computer besitzt sind vorbei. Die Meisten besitzen neben ihrem Smartphone auch ein Tablet. Inzwischen kommt bei vielen zusätzlich noch die Smartwatch hinzu, so dass jeder Einzelne gleich mehrere Computer besitzt.

Neu am sogenannten Wearable-Computing ist, dass die Geräte aktiv beim Tragen genutzt werden. Als verlängertes Display für das Smartphone machen sie vor allem durch Ereignisse, wie beispielsweise eingehende Nachrichten, auf sich aufmerksam. 

\subsection{Art der Mobilität}
Unser Projekt beschäftigt sich hauptsächlich mit der Mobilität von Endgeräten. In unserem Szenario bewegt sich der Anwender, ausgestattet mit einer Smartwatch, durch sein Haus und wird dabei aktiv über den Wechsel der Räumlichkeiten informiert. Dabei werden ihm im Vorfeld hinterlegte Notizen angezeigt, die er während seiner Bewegung wahrnimmt. Damit fällt unser Projekt in den Bereich der kontinuierlichen physikalischen Bewegung.

\subsection{Abstraktionen}
In unserem Modell beschreibt die Lokation den Raum, in dem sich die Smartwatch aktuell befindet. Zur Ausmessung der Räumlichkeiten ist das Smartphone erforderlich, während zur späteren Erkennung eines Raumwechsels nur die Smartwatch benötigt wird. Damit ist der Kontext lediglich davon abhängig, ob der Anwender neben seiner Smartwatch zusätzlich sein Smartphone dabei hat.

Die Smartwatch soll in der Lage sein, einen Raumwechsel automatisch zu erkennen. Da die Uhr über nur wenig Rechenleistung verfügt, muss der entsprechende Algorithmus möglichst ressourcenschonend arbeiten. Daher erscheint es sinnvoll die Berechnung des mathematischen Modells in einen Web-Service auszulagern und über eine entsprechende Middleware anzusprechen.

Um die Offline-Problematik zu umgehen, soll die Anwendung auf der Smartwatch auch ohne eine bestehende Internetverbindung in der Lage sein einen Raumwechsel zu erkennen. Eine weitere Herausforderung neben der Offline-Problematik ist, aufgrund unterschiedlicher Displayformen und -größen, das Thema Design und Usability.

\subsection{Mobiles Dilemma}
Um das Mobile Dilemma zu vermeiden soll unsere Kernfunktionalität auch ohne bestehende Internetverbindung zur Verfügung stehen. Lediglich für die aufwändige Berechnung des mathematischen Modells, die insgesamt nur einmal vorgenommen werden muss, benötigt der Anwender sein Smartphone mit einer Verbindung zum Internet. Auf der Smartwatch, die tendenziell eher ein schwaches Gerät ist, wollen wir einen möglichst ressourcenschonenden Algorithmus verwenden. Damit ist unsere Anwendung einzig bei der Einrichtung abhängig vom Internet. Im normalen Betrieb kommen wir jedoch nicht in die Situation, in der das Verhalten der Anwendung speziell an die mobile Situation adaptiert werden müsste.


