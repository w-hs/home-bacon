\section{Einordnung in Mobile-Computing}
Bei der Einordnung unseres Projektes in den Themenkomplex Mobile-Computing sind vor allem die Bereiche Ubiquitous- und Wearable-Computing relevant. 

In der heutigen Zeit besitzt nahezu jeder Mensch ein Smartphone. Die intelligenten Telefone sind allgegenwärtig und greifen permanent, von überall, auf Daten und Dienste aus dem Internet zurück. Die Zeiten in denen eine Person nur einen Computer besitzt sind vorbei. Neben dem PC verfügt man zusätzlich über ein Smartphone. Die Meisten besitzen darüber hinaus noch ein Tablet und inzwischen kommen auch immer mehr Smartwatches hinzu.

Neu am sogenannten Wearable-Computing ist, dass die Geräte aktiv beim Tragen genutzt werden. Als verlängertes Display für das Smartphone machen sie vor allem durch Ereignisse, wie beispielsweise eingehende Nachrichten, auf sich aufmerksam. 

\subsection{Art der Mobilität}
Unser Projekt beschäftigt sich hauptsächlich mit der Mobilität von Endgeräten. In unserem Szenario bewegt sich der Anwender, ausgestattet mit einer Smartwatch, durch sein Zuhause und wird dabei aktiv über den Wechsel der Räumlichkeiten informiert. Dabei werden ihm im Vorfeld hinterlegte Notizen angezeigt, die er dann während seiner Bewegung wahrnimmt. Damit fällt unser Projekt in den Bereich der kontinuierlichen physikalischen Bewegung.

\subsection{Abstraktionen}
- Modell: Lokation und Kontext beschreiben
\\- Algorithmen: Vorwiegend zur Lokation und ggf. für Ausführung auf schwacher Smart Watch
\\- Middleware: Brauchen wir ein Backend? Ggf. gar nicht notwendig
\\- Anwendungen: Anwendung auf Smart Watch, Offline-Fähigkeit, Lokale und globale Vernetzung

\subsection{Mobiles Dilemma}
- Kernfunktionalität auch Offline (In-Door-Lokalisation + Notizen)
\\- Online könnte die Synchronisierung mit anderen Geräten bringen


aufwendige Modellberechnungen die nicht auf der Uhr passieren sollen...

Berechnung nur einmal vornehmen, man möchte ja unabhängig vom Internet sein! Daher nur einmal Zugriff auf das Web nötig.