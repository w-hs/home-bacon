\section{Team Home-Bacon und die Suche nach dem verlorenen Beacon}

Es war 14:27 Uhr im Seminarraum A1.1.11, als wir den ersten großen Durchbruch
im Bereich der Raumerkennung erreicht hatten. Kurz zuvor hatten wir drei Beacons
im Raum und ein Beacon im Flur der Hochschule platziert. Nach dem jeweils
zwei- bis dreiminütigen Vermessen beider Räume war alles für den ersten
Feldversuch vorbereitet.

Die ersten Notizen wurden im Seminarraum erfolgreich hinterlegt. Die spannendste
Frage war jedoch, ob das erstellte Modell tatsächlich den Raumwechsel zum Flur 
bemerken würde. Also öffne ich die Tür und gehe drei Schritte aus dem Raum heraus.
Sofort meldet mir die Uhr den Raumwechsel in den Flur als Toast-Nachricht. Das 
Ergebnis war schon Grund genug zum Feiern. Allerdings waren noch einige Dinge
zu tun. Beim Raumwechsel müssen die hinterlegten Nachrichten angezeigt werden.
Außerdem soll die Toast-Nachricht noch durch ein Vibrationssignal ersetzt werden.

Also konzentrierten wir uns wieder auf die Implementierung der fehlenden
Feautres und die Beseitigung von Fehlern. Den Raumwechsel haben wir dabei einige
Male ausprobiert, bis ich schockiert feststellen musste, dass der Flur nicht
mehr erkannt wurde. Waren alle vorherigen Tests nur zufällig korrekt gelaufen?
Spielen hier unerwartete Interferenzen mit anderen Bluetooth-Geräten eine Rolle?
Nein! Plötzlich stellte ich fest, dass der grüne G-Tag, den ich zuvor auf der
metallischen Fensterbank platziert hatte, verschwunden war.

Der Worst-Case war eingetreten. Wir hatten teamintern schon öfter Witze darüber
gemacht, was wir tun würden, falls uns jemand einen Beacon klauen würde.
In dem Fall würden wir den Dieb einfach mithilfe unserer App verfolgen.
Das war jetzt natürlich nicht möglich, da unsere App nur Räume erkennt.
Glücklicherweise hatten wir alle jedoch zu Testzwecken bereits Apps zur Erkennung
beliebiger Bluetooth-Geräte installiert. Die Adresse des gestohlenen Beacons
war schnell identifiziert und wir konnten im Flur noch ein schwaches Signal
messen. Der Dieb war noch in Reichweite!

Sofort schwärmte unser Team bewaffnet mit jeweils einem Bluetooth-Scanner aus.
Das Signal verschwand relativ schnell wieder, sobald wir uns von der ursprünglichen
Position des Beacons entfernten. Hatte jemand den Tag aus dem Fenster geworfen?
Die Gruppe, die sich häuslich neben Hörsaal II eingerichtet hatte, wirkte auf uns
sehr verdächtig. Da das Signal in der Nähe der Toiletten am stärksten war,
untersuchten wir auch zumindest das Männer- und das Behinderten-WC. Dabei haben wir
uns ziemlich deutlich über den gestohlenen Beacon unterhalten. Auch die
wohnlich eingerichtete Gruppe haben wir nach dem Verbleib des Beacons gefragt.
Es war ein nicht zu lösendes Rätsel. Der Beacon konnte noch von den Apps gemessen
werden, aber zu finden war er nicht.

Letztendlich war das Signal komplett erloschen. Frustration und Ärgernis vermischten 
sich zu einer Panik. Offenbar konnte der Dieb samt Beacon entkommen. Zu diesem Zeitpunkt
mussten wir uns wohl oder übel mit dem Gedanken abfinden, dass der Dieb gerade die Hochschule
verlässt und der Beacon für immer verloren ist. Kurz bevor wir aufgeben wollten,
kehrten wir noch einmal an den Ort mit dem zuletzt stärksten Bluetooth-Signal zurück:
Die Herren-Toilette. Und plötzlich tauchte das Signal unseres vermissten Beacons wieder
auf. Hatte etwa jemand den Beacon in den Mülleimer geworfen oder war derjenige tatsächlich
so dreist den Beacon die Toilette herunter zu spülen?

In letzter Minute hatten wir die entscheidende Idee und riskierten noch einen Blick hinter 
die zweite Tür. Tatsächlich, unschuldig lag der grüne Beacon unter der Heizung. Endlich 
hatten wir den verlorenen Beacon gefunden. Vermutlich hatte jemand versucht, ihn zu stehlen, 
ist aber dann durch unsere ausgiebige Suchaktion in Panik geraten. Oder hat sich hier ein 
Mitglied des Teams einen Spaß erlaubt? Das Vertrauen innerhalb des Teams konnte bis heute nicht
wiederhergestellt werden.

