\section{Team Home-Bacon und die Suche nach dem verlorenen Beacon}

Es war 14:27 Uhr im Seminarraum A1.1.11, als wir den ersten großen Durchbruch
im Bereich der Raumerkennung erreicht hatten. Kurz zuvor hatten wir drei Beacons
im Raum und ein Beacon im Flur der Hochschule platziert. Nach dem dem jeweils
zwei- bis dreiminütigen Vermessen beider Räume war alles für die den ersten
Feldversuch vorbereitet.

Die ersten Notizen wurden im Seminarraum erfolgreich hinterlegt. Aber die
spannendste Frage war, ob das erstellte Modell den Raumwechsel auf den
Flur bemerken würde. Ich öffne also die Tür und gehe drei Schritte aus dem Raum
heraus. Sofort meldet die Uhr den Raumwechsel in den Flur als Toast-Nachricht.
Das Ergebnis war schon Grund genug zum Feiern. Allerdings waren noch einige Dinge
zu tun. Beim Raumwechsel müssen die hinterlegten Nachrichten angezeigt werden.
Außerdem soll die Toast-Nachricht noch durch ein Vibrationssignal ersetzt werden.

Also konzentrierten wir uns wieder auf die Implementierung der fehlenden
Feautres und die Beseitigung von Fehlern. Den Raumwechsel haben wir dabei einige
Male ausprobiert, bis ich schockiert feststellen musste, dass der Flur nicht
mehr erkannt wurde. Waren alle vorherigen Tests nur zufällig korrekt gelaufen?
Spielen hier unerwartete Interferenzen mit anderen Bluetooth-Geräten eine Rolle?
Doch plötzlich stellte ich fest, dass der grüne G-Tag, der zuvor auf der metallischen
Fensterbank platziert war, nicht mehr dort lag.

Der Worst-Case war eingetreten. Wir hatten teamintern schon öfter Witze darüber
gemacht, was wir tun würden, falls uns jemand einen Beacon klauen würde.
In dem Fall würden wir den Dieb einfach mithilfe unserer App verfolgen.
Das war jetzt natürlich nicht möglich, da unsere App nur Räume erkennt.
Allerdings hatten wir alle zu Testzwecken bereits Apps zur Erkennung von
beliebigen Bluetooth-Geräten installiert. Die Adresse des fehlenden Beacons
war schnell identifiziert und wir konnten im Flur noch ein schwaches Signal
messen. Der Dieb war noch in Reichweite!

Unser Team schwärmte sofort bewaffnet mit jeweils einem Bluetooth-Scanner aus.
Das Signal verschwand relativ schnell wieder, sobald wir uns von der ursprünglichen
Position des Beacons entfernten. Hatte jemand den Tag aus dem Fenster geworfen?
Die Gruppe die sich häuslich neben Hörsaal II eingerichtet hatten wirkten auch
sehr verdächtig. Da das Signal in der Nähe der Toiletten am stärksten war,
untersuchten wir auch zumindest das Männer- und das Behinderten-WC. Dabei haben wir
uns ziemlich deutlich über den gestohlenen Beacon unterhalten. Auch die
wohnlich eingerichtete Gruppe haben wir nach dem Verbleib des Beacon gefragt.
Es war ein nicht zu lösendes Rätsel. Der Beacon konnte noch von den Apps gemessen
werden, aber zu finden war er nicht.

Wir wollten schon aufgeben, als das Signal komplett verschwand. Der Dieb
war entkommen. Oder zumindest mussten wir zu diesem Zeitpunkt davon ausgehen.
Vielleicht verlässt er gerade die Hochschule und wir werden den Beacon nie
wieder sehen. Im letzten Versuch sind wir an den Ort mit dem bisher stärksten
Signal zurückgekehrt: Das Männer-WC. Und da tauchte der Beacon wieder auf.
Hat jemand den Beacon in den Mülleimer neben dem Waschbecken geworfen? Oder
wurde sogar versucht den Beacon die Toilette runter zu spülen und er schwimmt
jetzt in einer Schüsseln?

Gott sei Dank hat Dennis direkt hinter die zweite Tür geschaut. Und da lag
der grüne Tag unschuldig unter der Heizung. Endlich hatten wir den verlorenen
Beacon gefunden! Jemand hatte wohl versucht, den Beacon zu stehlen, ist aber
dann in Panik geraten, als wir mit unseren Scanner wild durch die Gegend
gerannt sind. 
