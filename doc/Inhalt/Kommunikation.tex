\section{Kommunikation}
Wie im Kapitel \ref{sec:architektur} beschrieben, besteht unsere Anwendung aus drei Komponenten, die untereinander kommunizieren müssen. Das Smartphone spricht auf der einen Seite mit dem Web-Service zur Berechnung des Models, auf dem die Lokalisierung beruht. Auf der anderen Seite sendet es Daten, wie z.B. eine Notiz, und Steuerbefehle an die Smartwatch, um z.B. Bluetooth-Scans zu starten oder zu stoppen. Umgekehrt sendet die Smartwatch Daten an das Smartphone, z.B. die Ergebnisse eines Scans. Die verschiedenen Kommunikationswege und ihre Nutzung werden im Folgenden näher beschrieben.

\subsection{Zwischen Smartphone und Web-Service}
@Fabian
-via http
-serialisierte Scanergebnisse (aufgeteilt in Blöcke?)


\subsection{Zwischen Smartphone und Smartwatch}
Die Kommunikation zwischen dem Smartphone und der Smartwatch muss bidirektional sein, um die benötigten Funktionalitäten erfüllen zu können. Die folgenden Schnittstellendefinitionen beschreiben diese.

XXX Interfaces Abbildungen XXX


- GoogleAPI
	- DataItems getestet -> Synchronisiert  unplanbar
	- Messages mit serialisierten Daten im Byte[]

-Beispiel messen

\subsection{Zwischen Service und Activity}
-Service muss Infos weiterreichen
-Service muss Activity starten
-Beispiel Note

