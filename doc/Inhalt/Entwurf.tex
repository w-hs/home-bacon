\section{Entwurf}
In diesem Kapitel stellen wir die Plattformen vor, auf die wir uns bei der Entwicklung beschränken wollen. Anschließend skizzieren wir unser System mithilfe eines Diagramms. Im letzten Abschnitt gehen wir auf die wichtigsten Aspekte beim Design der Smartwatch-Anwendung ein.

\subsection{Plattformen}
In diesem Projekt geht es im Wesentlichen um zwei Aspekte. Zum einen wollen wir uns mit der Anwendungsentwicklung auf Smartwatches beschäftigen, zum anderen wollen wir einen neuen Ansatz zur Lokalisation in Gebäuden ausprobieren. Aus diesem Grund verzichten wir auf aufwendige hybride Lösungen und beschränken uns auf die Entwicklung zweier nativer Applikationen für die Android- bzw. Android-Wear-Plattform. Für die Lokalisation innerhalb des Gebäudes wollen wir Bluetooth-Beacons einsetzen. Bezüglich der Beacons haben wir uns, vor allem aufgrund des niedrigen Stromverbrauchs, auf den Standard Bluetooth Low Energy festgelegt.

\subsection{Systemübersicht}
\label{sec:systemuebersicht}
Abbildung \ref{fig:Übersicht} zeigt eine grundlegende Übersicht unseres Systems:

\begin{figure}[H]
\centering
\includegraphics[width=0.95\linewidth]{Bilder/Uebersicht}
\caption{Systemübersicht}
\label{fig:Übersicht}
\end{figure}

Im ersten Schritt, noch vor Inbetriebnahme unseres Systems, müssen die Räume des Gebäudes mit Bluetooth-Beacons bestückt werden. Diese senden kontinuierlich ein Signal, das durch die Smartwatch-App empfangen wird. Bei der Inbetriebnahme des Systems muss zunächst jeder Raum des Gebäudes vermessen werden. Ein entsprechender Scan lässt sich über die Smartphone-App steuern. Nachdem das Gebäude vollständig vermessen wurde, muss im nächsten Schritt das mathematische Modell zur Erkennung des aktuellen Raums berechnet werden. Da diese Berechnung in einen Web-Service ausgelagert ist, wird an dieser Stelle einmalig eine Internetverbindung vorausgesetzt. Nach erfolgreicher Berechnung ist das System betriebsbereit und erkennt von nun an automatisch das Betreten bzw. Verlassen eines Raums.

Nachdem das System initial eingerichtet wurde, benötigt der Anwender sein Smartphone nur noch, um Notizen im aktuellen Raum zu hinterlegen. Über die Smartphone-App lassen sich Notizen zusätzlich mit den Ereignissen Verlassen oder Betreten des Raums verknüpfen. Die Daten werden nach der Eingabe an die Smartwatch geschickt und dort gespeichert. Damit wird das Smartphone für den aktiven Systembetrieb nicht mehr benötigt.

\subsection{Einschränkungen durch die Smartwatch}
\label{sec:einschraenkungen}
Grundsätzlich verfügen Smartwatches über die gleichen Eingabemöglichkeiten wie Smartphones. Durch das kleine Display entfällt jedoch die Möglichkeit einer vollständigen Tastatur, so dass Benutzerinteraktionen häufig über Wischgesten oder sogenannte Action-Buttons realisiert werden. Action-Buttons sind vom System bereitgestellte Buttons mit einem Icon und einem beschreibenden Text. Sie werden im Vollbild angezeigt und dienen der Annahme von Benutzerinteraktionen.

Eine weitere Besonderheit ist neben der Größe auch die Form des Displays. Damit die Applikation auf Uhren mit eckigen und rundem Display jeweils gleich angezeigt werden, bietet Android-Wear spezielle Layouts, welche auch auf einem runden Display immer den größtmöglichen eckigen Bereich zur Anzeige nutzen.

Darüber hinaus besitzen nicht alle Smartwatches die Möglichkeit sich mit einem WLAN zu verbinden. Daher ist ein direkter Zugriff auf das Internet aus der Smartwatch-App nicht immer möglich.
